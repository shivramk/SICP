\documentclass[a4paper,12pt]{article}
\usepackage{amsmath,amsthm,amssymb,listings}
\begin{document}

Let each interval $i$ be represented by the pair $(l, u)$, where $l$ is the lower bound and $u$ is the upper bound. \\

Let the weight be $w = \frac{(u-l)}{2}$ \\

Consider the addition of two interval $i_1$ and $i_2$
\begin{align}
i_1+i_2 &= (l_1+l_2, u_1+u_2) \\
w & = \frac{u_1 + u_2 - (l_1 + l_2)}{2} \\ 
  & = \frac{u_1 - l1}{2} + \frac{u_2 - l_2}{2} \\
  & = w_1 + w_2
\end{align}

Now consider subtraction
\begin{align}
i_1-i_2 &= (l_1-u_2, u_1-l_2) \\
w & = \frac{u_1 - l_2 - (l_1 - u_2)}{2} \\ 
  & = \frac{u_1 - l1}{2} + \frac{u_2 - l_2}{2} \\
  & = w_1 + w_2
\end{align}

This shows that for addition and subtraction the width of the result is a function of the widths of the arguments. The same is not the case for multiplication and division. 

For multiplication consider the two intervals $(3, 4)$ and $(5, 6)$, each with a width of $0.5$. The product of the two intervals is $(15, 24)$ which has a width of $4.5$, which is no way a function of the widths of the arguments alone.

For division consider the same interval again. The result is $(0.5, 0.8)$ which has a width of $0.15$. Again the resultant width is not a function of the widths of the arguments alone.

\end{document}