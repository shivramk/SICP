\documentclass[a4paper,12pt]{article}
\usepackage{amsmath,amsthm,amssymb}
\begin{document}

Answering this problem in the general case is difficult. We will consider two separate cases
\begin{enumerate}
\item The case when the huffman tree is balanced
\item The special case where the relative frequenies of the $n$ symbols are as described in \emph{Exercise 2.71}
\end{enumerate}

\bigskip \emph{Case 1}
In this case given that the huffman tree has $n$ symbols the height of the tree will be given by $\log_2(n)$. In the root node all the $n$ symbols will be present. The number of symbols at depth 2 will be $n/2$, at depth 2, $n/4$, at depth 3, $n/8$, and so on until we reach the leaf which will have 1 symbol. 

The encoding procedure starts from the root of the tree and works downwards towards the leaft. At each node it must search through the symbols both branches and decide which branch to take. In the worst case the encoding procedure will first have to search through $n$ symbols, then $n/2$, then $n/4$ ... till it reaches the leaf. In the average case it will be $n/2$, then $n/4$ ... till it reaches the leaf. 

This can be written as
\begin{align*}
O & = \frac{n}{2} + \frac{n}{4} + \frac{n}{8} + ... + 1 \\
  & = n (\frac{1}{2} + \frac{1}{4} + \frac{1}{8} + ... )
\intertext{Simplifying the geometric progression}
  & = n (\frac{1 - 2^n}{1-\frac{1}{2}} - 1) \\
  & = n (\frac{1}{\frac{1}{2}} - 1) \\
  & = O(n)
\end{align*}

This gives us a growth that is linear in the number of symbols

\bigskip \emph{Case 2} This is more of degenerate case. At the root node we have $n$ symbols. As we descend left we have $n-1$, then $n-2$ ... till we reach the leaves which have 1 symbol. This tree has the nice property that if we are looking of an element at depth $k$ then it will be present at position $n-k$ in the symbols list. And this position doesn't change as we traverse down the tree. Therefore the order of growth is
\begin{align*}
O &= k(n-k)
\intertext{Which will be maximum when $k = n/2$}
  &= \frac{n}{2}(n - \frac{n}{2}) \\
  &= \frac{n^2}{4} \\
  &= O(n^2)
\end{align*}

The growth will be $O(n^2)$ in the average case as well

\end{document}