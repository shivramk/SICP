\documentclass[a4paper,12pt]{article}
\usepackage{amsmath,amsthm,amssymb,listings}
\begin{document}

\noindent Consider the given code

\lstset{
language=Lisp,                            
basicstyle=\ttfamily,  
}

\begin{lstlisting}
(define (expmod base exp m)
 (cond ((= exp 0) 1)
  ((even? exp) (remainder 
      (* (expmod base (/ exp 2) m)
         (expmod base (/ exp 2) m)) m))
  (else (remainder 
      (* base (expmod base (- exp 1) m)) m))))
\end{lstlisting}

\bigskip\noindent The problem here is the multiplication of \texttt{(expmod base (/ exp 2) m)} by itself.
If the square function were being used, the expression \texttt{(expmod base (/ exp 2) m)}
would be evaluated only once. But now it is being evaluated twice. This changes
the complexity from $O(log(n))$ to $O(n)$

\bigskip\noindent Proof by example

1. In the base case where exp = 0
\begin{align}
steps(0) &= 1 
\end{align}

2. When exp = 1
\begin{align}
steps(1) &= 2
\end{align}

3. For exp = 2
\begin{align}
steps(2) &= 2 * steps(1) \\
& = 2 * 2 \\
& = 4
\end{align}
                  
4. For exp = n
\begin{align}
steps &= 2 * (steps(n-1)) \\
& = 2 * n \\
& = 2n
\end{align}

Hence the order of growth is $\Theta(n)$

\bigskip\noindent Proof by master method

1. The recurrence relation for the order of growth of this process can be
written as
\begin{align}T(n) = aT(n/b) + f(n)\end{align}

2. In this case (where $a=2, b=2$ and $f(n)=k$ where k is a constant)
\begin{align}T(n) = 2T(n/2) + k\end{align}

3. The master method says that if
               $f(n) = O(n^{(\log_b{a}-\epsilon)})$ for some constant $\epsilon > 0$ then
               $T(n) = \Theta(n^{\log_b{a}})$
               
4. Subtituting the values of a and b we get
\begin{align}
T(n) &= \Theta(n^{\log_b{a}}) \\
& = \Theta(n^{\log_2{2}}) \\
& = \Theta(n^1) \\
& = \Theta(n)
\end{align}

\end{document}