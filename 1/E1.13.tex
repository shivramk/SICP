\documentclass[a4paper,12pt]{article}
\usepackage{amsmath,amsthm,amssymb}
\begin{document}

Let $\phi=\frac{1+\sqrt{5}}{2}$ and $\psi=\frac{1-\sqrt{5}}{2}$. We will prove that \( Fib(n) = \frac{\phi^n - \psi^n}{\sqrt{5}} \) using mathematical induction. \\

1. In the base case
\begin{align}
Fib (n) &= \frac{\phi - \psi}{\sqrt{5}} \\ 
& = \frac{\frac{1+\sqrt{5}}{2} - \frac{1-\sqrt{5}}{2}}{\sqrt{5}} \\
& = \frac{1+\sqrt{5}-1+\sqrt{5}}{2\sqrt{5}} \\
& = 1
\end{align}

2. Lets assume that \(Fib(k) = \frac{\phi^k - \psi^k}{\sqrt{5}} \) \\

3. Consider \(Fib(k) + Fib(k-1) \)
\begin{align}
Fib(k) + Fib(k-1) &=  \frac{\phi^k - \psi^k}{\sqrt{5}} + \frac{\phi^{k-1} - \psi^{k-1}}{\sqrt{5}} \\
& = \frac{1}{\sqrt{5}} (\phi^{k+1}(\frac{1}{\phi} + \frac{1}{\phi^2}) - \psi^{k+1}(\frac{1}{\psi} + \frac{1}{\psi^2}))
\end{align}

We know that $\phi$ and $\psi$ are the roots of the equation \(x^2 - x - 1 = 0 \), which means
\begin{align}
\phi^2 - \phi - 1 &= 0 \\
\phi + 1 = \phi^2 \\
\frac{1}{\phi} + \frac{1}{\phi^2} = 1 
\end{align}

Similarly \( \frac{1}{\psi} + \frac{1}{\psi^2} = 1 \). Substituting in $(6)$ gives us
\begin{align}
Fib(k) + Fib(k-1) &= \frac{1}{\sqrt{5}} (\phi^{k+1} - \psi^{k+1}) \\
& = Fib(k+1)
\end{align}

For larger values of n the second term \( \psi^n \) can be ignored. Which gives us 
\begin{align}
Fib(n) &= \frac{\phi^n}{\sqrt{5}}
\end{align}
\end{document}